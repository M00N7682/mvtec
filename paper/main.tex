\documentclass[11pt]{article}
\usepackage[margin=1in]{geometry}
\usepackage{amsmath,amssymb}
\usepackage{graphicx}
\usepackage{booktabs}
\usepackage{hyperref}
\usepackage{caption}
\usepackage{subcaption}
\usepackage{kotex}

\title{RDI-Net: 극단적 클래스 불균형 환경에서의 결함 삽입 기반 이미지 합성 및 분류기 기반 이상 탐지}
\author{문덕룡}
\date{\today}

\begin{document}
\maketitle

\begin{abstract}
산업 결함 데이터는 정상 샘플이 풍부한 반면 결함 샘플은 극히 희소하여, 데이터 불균형이 학습 성능과 일반화에 구조적 제약으로 작용한다.
이러한 환경에서 합성 데이터는 유망한 대안이지만, 단순한 복제나 나이브한 붙여넣기 방식은 경계 인공물(artifact)을 유발하거나 결함 패턴의 구조를 왜곡하여 다운스트림 탐지 효용을 떨어뜨릴 수 있다.
본 연구는 정상 이미지의 배경 문맥을 최대한 보존하면서 결함을 국소적으로 삽입하는 합성 프레임워크 \textit{RDI-Net}을 제안한다.
핵심 아이디어는 전체 이미지를 새로 생성하는 대신, 결함 마스크 영역에서만 잔차(residual)와 블렌딩 맵을 생성하여 결함을 ``삽입하고 조화(harmonize)''하는 것이다.
우리는 MVTec AD의 다섯 개 카테고리(bottle, cable, capsule, hazelnut, metal\_nut)에서 결함 K-shot 설정($K\in\{1,5,10\}$)을 구성하고, ResNet-18 기반 이진 분류기에서 image-level AUROC로 효용을 평가한다.
또한 합성 데이터가 ``너무 티가 나서'' 모델이 합성 인공물에 과적합하는 위험을 통제하기 위해, 실제 결함 패치와 합성 결함 패치의 구분 가능성(real-vs-synthetic discriminability)을 별도의 편향 점검(bias check)으로 보고한다.
\end{abstract}

\section{Introduction}
산업 현장에서의 비전 기반 검사(visual inspection)는 결함을 조기에 탐지해 품질과 안전을 보장하는 핵심 공정이다.
그러나 실제 데이터 수집 과정에서는 정상 샘플이 대량으로 축적되는 반면, 결함은 발생 빈도 자체가 낮고 다양한 유형으로 분산되기 때문에 라벨이 달린 결함 데이터를 충분히 확보하기 어렵다.
이와 같은 극단적 클래스 불균형은 지도 학습 기반 분류기에서 특히 치명적이며, 결함 클래스의 다양성을 학습하지 못해 결정경계가 불안정해지거나 특정 결함 패턴에 과적합되는 문제가 발생한다.

이러한 맥락에서 합성 데이터는 결함 샘플의 ``절대량''을 보완할 수 있는 실용적 수단이지만, 합성의 품질을 무엇으로 정의할 것인지는 단순하지 않다.
예를 들어, 정상 이미지 위에 결함 패치를 단순히 붙여넣는 방식은 구현이 쉽지만, 결함 경계 주변에 비자연스러운 블렌딩 흔적이 남기 쉽고, 모델이 결함 자체가 아니라 합성 흔적을 단서로 삼아 과적합하는 위험이 있다.
반대로 전체 이미지를 생성하는 대규모 생성 모델은 데이터와 계산 요구가 크며, 희소 모드를 안정적으로 학습시키기 어렵다.

본 연구는 ``불균형 환경에서 유용한 결함 합성''을 목표로, 결함을 전역적으로 생성하는 대신 국소 삽입 문제로 재정의한다.
구체적으로 RDI-Net은 정상 배경을 최대한 유지한 채 결함 영역에서만 잔차 및 블렌딩 맵을 예측하여 합성 결함을 생성한다.
또한 실험 프로토콜을 재현 가능하게 구성하기 위해 MVTec AD에서 K-shot 결함 설정($K\in\{1,5,10\}$)을 만들고, 분류기 기반 이상 탐지에서 image-level AUROC를 주요 지표로 사용한다.
마지막으로 합성 데이터가 만들어내는 잠재적 편향을 점검하기 위해, 실제 결함 패치와 합성 결함 패치가 얼마나 쉽게 구분되는지를 별도 지표로 보고한다.

\section{Problem Setup}
정상 이미지를 $x\in\mathbb{R}^{H\times W\times 3}$, 결함 마스크를 $m\in\{0,1\}^{H\times W}$로 두자.
훈련 시점에서 결함 샘플은 카테고리별로 $K$개만 주어진다고 가정하며($K$-shot), 이들 결함과 마스크를 이용해 추가적인 합성 결함 이미지 $\tilde{x}$를 생성한다.
다운스트림 과제는 정상/결함 이진 분류기에 기반한 이상 탐지이며, 테스트 셋에서 image-level AUROC를 최대화하는 것을 목표로 한다.

\section{Method}
\subsection{Baselines}
베이스라인으로는 (i) 결함 패치를 정상 이미지 위에 직접 삽입하는 Copy-Paste와, (ii) 마스크 경계를 완화하여 삽입 흔적을 줄이려는 간단한 블렌딩 변형(poisson-like)을 구현한다.
이때 결함 패치는 K-shot 결함 이미지와 마스크로부터 추출한 패치 뱅크에서 샘플링하며, 각 정상 이미지에 결함을 1개만 삽입하도록 고정해 합성량에 의한 효과를 최소화한다.

\subsection{RDI-Net (to be completed)}
RDI-Net은 정상 이미지 $x$, 결함 마스크 $m$, 결함 프로토타입 패치 $p$를 입력으로 받아 잔차 $r$과 블렌딩 맵 $\alpha$를 출력한다.
최종 합성 이미지는 다음과 같이 정의한다.
\[
\tilde{x} = x + \alpha \odot m \odot r.
\]
여기서 $\odot$는 원소별 곱이다.
설계 의도는 명확하다. 마스크 외부에서는 $x$를 거의 변경하지 않으면서, 마스크 내부에서는 결함 패턴을 생성하고 경계 부근에서는 $\alpha$를 통해 자연스럽게 조화시킨다.
학습은 (i) 정상 영역 보존(Identity) 항, (ii) 경계 매끄러움(seam/TV) 항, (iii) 결함 패치 수준의 현실감을 위한 약한 적대적 손실, (iv) (선택적으로) 사전학습 인코더 임베딩에서 결함 영역의 통계(모멘트/공분산)를 보존하는 항으로 구성한다.
이러한 구성은 불균형 환경에서 결함 모드의 붕괴를 완화하면서도, 결함-배경 간 문맥 정합성을 유지하는 것을 목표로 한다.

\section{Experiments}
\subsection{Dataset and Protocol}
우리는 MVTec AD의 다섯 개 카테고리(bottle, cable, capsule, hazelnut, metal\_nut)를 사용한다.
해상도는 256으로 고정한다.
K-shot 결함 프로토콜은 재현성을 위해 고정 시드로 구성하며, 테스트 셋은 각 카테고리의 표준 테스트 구조를 유지한다.

\subsection{Metrics}
주요 지표는 테스트 셋에서의 image-level AUROC이다.
또한 합성 편향을 점검하기 위해, 실제 결함 패치와 합성 결함 패치를 구분하는 이진 분류기의 AUROC를 함께 보고한다.
이 값이 지나치게 높다면(=매우 쉽게 구분된다면) 합성 인공물이 강하다는 신호로 해석할 수 있다.

\subsection{Results}
표~\ref{tab:main}은 파이프라인 출력(\texttt{outputs/baselines/summary.csv})으로부터 채워진다.

\begin{table}[t]
\centering
\caption{Image-level AUROC (5개 카테고리 평균). 파이프라인 결과로 채움.}
\label{tab:main}
\begin{tabular}{lccc}
\toprule
Method & $K=1$ & $K=5$ & $K=10$ \\
\midrule
No synth & -- & -- & -- \\
Copy-Paste & -- & -- & -- \\
Poisson-like & -- & -- & -- \\
PaDiM-lite (normal-only) & -- & -- & -- \\
RDI-Net (ours) & -- & -- & -- \\
\bottomrule
\end{tabular}
\end{table}

그림~\ref{fig:auroc_k}는 \texttt{scripts/make\_figures.py}가 생성한 K에 따른 평균 AUROC 변화를 시각화한다.

\begin{figure}[t]
\centering
\includegraphics[width=0.7\linewidth]{../outputs/figures/auroc_vs_k_mean.png}
\caption{K-shot에 따른 평균 AUROC 변화(베이스라인).}
\label{fig:auroc_k}
\end{figure}

\section{Discussion}
본 절에서는 불균형이 심한 구간(특히 $K=1$)에서 합성 증강이 도움이 되는 조건과, 반대로 합성 인공물이 강해 모델이 잘못된 단서에 의존하게 되는 조건을 논의한다.
또한 real-vs-synthetic 편향 점검 결과를 통해, 단순히 AUROC를 올리는 합성이 아니라 ``탐지에 유용하면서도 자연스러운'' 합성이 무엇인지 해석한다.

\section{Conclusion}
본 연구는 K-shot 결함 환경에서 결함 삽입 기반 합성 데이터를 활용해 분류기 기반 이상 탐지 성능을 개선하는 문제를 다루었다.
우리는 재현 가능한 데이터 분할과 베이스라인을 포함한 end-to-end 파이프라인을 제공하고, RDI-Net을 통해 국소 결함 삽입 및 조화에 기반한 합성 전략을 제안한다.

\bibliographystyle{plain}
\bibliography{refs}

\end{document}


